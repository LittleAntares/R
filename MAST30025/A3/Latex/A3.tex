\documentclass[12 pt, a4paper]{article}
\usepackage[utf8]{inputenc}
\usepackage[left=2.5 cm,top=2.54 cm,right=2.5 cm,bottom=2.54 cm]{geometry}
\usepackage[flushleft]{threeparttable}
\usepackage{mathtools}
\usepackage{amssymb}
\usepackage{amsmath}
\title{Linear Statistical Models Assignment 3}
\date{}
\usepackage[dvipsnames]{xcolor}
\usepackage{graphicx,wrapfig}
\author{Kim Seang CHY}
\usepackage{xcolor}
\usepackage{lipsum}
\usepackage{mwe}

%Table Adjustment
\usepackage{tabularx}
\newcommand{\tabitem}{~~\llap{\textbullet}~~}
\newcommand{\A}{\textbf{A}}
\newcommand{\M}{\textbf{M}}
\newcommand{\B}{\boldsymbol\beta}
\newcommand{\z}{\textbf{z}}



\begin{document}

\maketitle


\noindent \textbf{Question 1:}	Let $A$ be an $n \times p$ matrix with $n \geq p$.\\

\noindent \textbf{a.} Show directly that $r(A^c A) = r(A)$.\\

\noindent Let $\A=
\left[
\begin{array}{c|c}
\M & A_{12}\\
\hline
A_{21}& A_{22}\\
\end{array}
\right] $  where \M \space is a square matrix with $r(\A) \times r(\A)$ and $r(\A)=a\leq p$. By definition \M \space is a matrix with $r(\M)=r(\A)$\\

\noindent Using theorem 6.2, we can find $\A^c$ such that it has the following partition:

\noindent $\A^c=
\left[
\begin{array}{c|c}
\M^{-1} & 0\\
\hline
0 & 0\\
\end{array}
\right] $ where $\M^{-1}$ is the inverse of \M.\\

\vspace{0.5cm}
\noindent Hence $\A^c\A=
\left[
\begin{array}{c|c}
I_a & \M^{-1}A_{12} \\
\hline
0 & 0\\
\end{array}
\right] $ \\

		

\noindent All column vectors in $\M^{-1}A_{1,2}$ is a linear combination of the independent column vectors  in the identity matrix $I_a$.    Thus $r(\A^c\A)=r(I_a)=r(\M)=a$. Since $r(\A)=r(\M)$, this implied $r(\A)= r(\A^c\A)$.\\




\noindent \textbf{b.} Show directly that $A^c A$ is idempotent.\\

\noindent Since $A^c$ is a conditional inverse for A then $AA^cA$=A. Thus,
\begin{align*}
\left(A^cA\right)^2&=A^cA A^cA \\
&= A^cA
\end{align*}		
 \noindent Hence, $A^cA$ is idempotent\\
		
		
\noindent \textbf{c.} Show directly that $A(A^TA)^c A^T$ is unique (invariant to the choice of conditional inverse). \\

	

\noindent Consider conditional inverse $(A^TA)^c_1$ and an arbitrary conditional inverse $(A^TA)^c_i$ where $i\neq 1$. Now using the properties $A=A(A^TA)^c_i(A^TA$) and $A^T= (A^TA)^c_1(A^TA)A^T.$, we get the following:
\begin{align*}
A(A^TA)^c_1A^T&= A(A^TA)^c_i(A^TA)(A^TA)^c_1A^T\\
&=A(A^TA)^c_iA^T
\end{align*}

\noindent Since,$ A(A^TA)^c_1A^T=A(A^TA)^c_iA^T$, this implied it is unique and invariant to the choice of conditional inverse.


\pagebreak

\noindent \textbf{Question 3:}\\
\vspace{0.5cm}

Let $ t= \left[
\begin{array}{c}
t_1 \\
\hline
t_2
\end{array}
\right]=
\left[
\begin{array}{c}
X_1^TX_1z_1 \\
\hline
X_2^TX_2z_2
\end{array}
\right] $ and $X= \left[
\begin{array}{c|c}
X_1 & X_2
\end{array}
\right]$ \

\vspace{0.5cm}
$X^TX\textbf{a}=
 \left[
\begin{array}{c}
X_1^T\\
\hline
X_2^T
\end{array}
\right]
 \left[
\begin{array}{c|c}
X_1 & X_2
\end{array}
\right]
\textbf{a}=
 \left[
\begin{array}{c|c}
X_1^TX_1 & X_1^TX_2\\
\hline
X_2^TX_1 &X_2^TX_2
\end{array}
\right]\textbf{a}
$\\

\vspace{0.5cm}
\noindent We can to rewrite the system of linear equation for  $X^TX\textbf{a}=t$,  as an augmented matrix form as $[X^TX|t]$.\\


$[X^TX|t]= \left[
\begin{array}{c c |c}
X_1^TX_1 & X_1^TX_2 &X_1^TX_1z_1 \\
X_2^TX_1 &X_2^TX_2 &X_2^TX_2z_2 \\ 
\end{array}
\right]=
\left[
\begin{array}{c c}
X_1^T & 0\\
0 &X_2^T\\ 
\end{array}
\right]
 \left[
\begin{array}{c c |c}
X_1& X_2&X_1z_1\\
X_1 &X_2&X_2z_2 \\ 
\end{array}
\right]$\\

\noindent Since, $X_2$ is continuous factor we can inferred that $X_2$ is column full rank and since $X_1$ less then full column rank we can inferred that $X_1$ can be written as a linear combination $X_2$. \\

\noindent Thus by theorem 6.2 that 
$r\left(\left[
\begin{array}{c c |c}
X_1& X_2&X_1z_1\\
X_1 &X_2&X_2z_2 \\ 
\end{array}
\right]\right)=r(X^TX)$ if and only if
$\left[
\begin{array}{c c |c}
X_1& X_2&X_1z_1\\
X_1 &X_2&X_2z_2 \\ 
\end{array}
\right]$ is a consistent system.\\

\noindent Thus by the fact that $X_2$ is a linear combination of $X_1$and then $\left[
\begin{array}{c c}
X_1^TX_1 & X_1^TX_2\\
X_2^TX_1 &X_2^TX_2\\ 
\end{array}
\right]$ are linear combination of $\left[
\begin{array}{c}
t_1 \\
\hline
t_2
\end{array}
\right] $ if there exist a $z_1$ such that $X_1^TX_1z_1=t_1$ is a consistent system. \\

\noindent Since, $t_1^T\beta_1$ is estimable then $X_1^TX_1z_1=t_1$ is a consistent system hence $X^TXz=t$ is also a consistent system. Thus, if $t_1^T\beta_1$ is estimable then $t^T\beta$ where $\beta^T=[\beta_1^T|\beta_2^T]$. 
 





\end{document}
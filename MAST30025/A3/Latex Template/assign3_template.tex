\documentclass[a4paper]{article}
\usepackage{amsmath,fullpage}

\newcommand{\bemat}{\left[\begin{array}}
\newcommand{\enmat}{\end{array}\right]}
\newcommand{\be}{\beta}
\newcommand{\B}{\boldsymbol\beta}
\newcommand{\e}{\varepsilon}
\newcommand{\E}{\boldsymbol\varepsilon}
\newcommand{\tb}{\mathbf{t}}
\newcommand{\y}{\mathbf{y}}
\newcommand{\z}{\mathbf{z}}
\newcommand{\0}{\mathbf{0}}


\begin{document}

\begin{centering}

\LARGE MAST30025 Linear Statistical Models

\vspace{0.5cm}

\Large Assignment 3, 2021

\end{centering}

\vspace{0.5cm}

\begin{enumerate}
	\item	Let $A$ be an $n \times p$ matrix with $n \geq p$.
	\begin{enumerate}
		\item Show directly that $r(A^c A) = r(A)$.
		\item Show directly that $A^c A$ is idempotent.
		\item Show directly that $A(A^TA)^c A^T$ is unique (invariant to the choice of conditional inverse).
	\end{enumerate}

\item We are interested in examining the lifetime of three different types of bulb (in 1000 hours). A study is conducted and the following data obtained:

\vspace{0.2cm}

\begin{tabular}{ccc}
\multicolumn{3}{c}{Bulb type} \\
1 & 2 & 3 \\
\hline
22 & 16 & 28 \\
23 & 18 & 27 \\
24 & 19 & 29 \\
22 &    & 29 \\
26
\end{tabular}

\vspace{0.2cm}

We fit the model
\[y_{ij} = \mu + \tau_i + \e_{ij},\]
where $\mu$ is the overall mean and $\tau_i$ is the effect of the $i$th type of bulb.

\vspace{0.2cm}

\textbf{For this question, you may NOT use the \texttt{lm} function in R.}

	\begin{enumerate}
	\item Find a conditional inverse for $X^TX$, using the algorithm given in Theorem 6.2.\label{condin}
	\item Find $s^2$.
	\item Is $\mu + 2\tau_1 + \tau_2$ estimable?
	\item Find a 90\% confidence interval for the lifetime of the 2nd type of bulb.
	\item Test the hypothesis that there is no difference in lifetime between 1st and 3rd types of bulb, at the 5\% significance level.
	\end{enumerate}



\item Consider a linear model with only categorical predictors, written in matrix form as $\y = X_1 \B_1 + \E_1$. Suppose we add some continuous predictors, resulting in an expanded model $\y = X \B + \E$. 
	
	Now consider a quantity $\tb^T\B$, where $\tb^T = [\tb_1^T|\tb_2^T]$ is partitioned according to the categorical and continuous predictors. Show that if $\tb_1^T\B_1$ is estimable in the first model, then $\tb^T\B$ is estimable in the second model.

	If you write $X = [X_1 | X_2]$, you may assume that $r(X) = r(X_1) + r(X_2)$.

	\vspace{0.2cm}

	\emph{Hint: Use Theorems 6.9 and 6.3. For any vectors $\z_1$ and $\z_2$, you can write \[\bemat{cc|c} X_1^T X_1 & X_1^T X_2 & X_1^T X_1 \z_1 \\ X_2^T X_1 & X_2^T X_2 & X_2^T X_2 \z_2 \enmat = \bemat{cc} X_1^T & \0 \\ \0 & X_2^T\enmat\bemat{cc|c} X_1 & X_2 & X_1 \z_1 \\ X_1 & X_2 & X_2 \z_2 \enmat.\]}



\item Data was collected on the world record times (in seconds) for the one-mile run. For males, the records are from the period 1861--1999, and for females, from the period 1967--1996. The data is given in the file \texttt{mile.csv}, available on the LMS.

\begin{enumerate}
	\item Plot the data, using different colours and/or symbols for male and female records. Without drawing diagnostic plots, do you think that this data satisfies the assumptions of the linear model? Why or why not?
	\item Test the hypothesis that there is no interaction between the two predictor variables. Interpret the result in the context of the study.
	\item Write down the final fitted models for the male and female records. Add lines corresponding to these models to your plot from part (a).
	\item Calculate a point estimate for the year when the female world record will equal the male world record. Do you expect this estimate to be accurate? Why or why not?
	\item Is the year when the female world record will equal the male world record an estimable quantity? Is your answer consistent with part (d)?
	\item Calculate a 95\% confidence interval for the amount by which the gap between the male and female world records narrow every year.
	\item Test the hypothesis that the male world record decreases by 0.4 seconds each year. 
\end{enumerate}


\item You wish to perform a study to compare 2 medical treatments (and a placebo) for a disease. Treatment 1 is an experimental new treatment, and costs \$5000 per person. Treatment 2 is a standard treatment, and costs \$2000 per person. Treatment 3 is a placebo, and costs \$1000 per person. You are given \$100,000 to complete the study. You wish to test if the treatments are effective, i.e., $H_0: \tau_1 = \tau_2 = \tau_3$.

	\begin{enumerate}
		\item Determine the optimal allocation of the number of units to assign to each treatment.
		\item Perform the random allocation. You must use R for randomisation and include your R commands and output.
	\end{enumerate}

\end{enumerate}

\end{document}
